\section{Lagrangian Dynamics}
\label{sec:lagrangian}
Articulated human motions can be described by a set of dynamic
equations of motion of multibody systems. Since the direct application
of Newton's second law becomes difficult when a complex human skeleton
is considered, we use \emph{Lagrange's equations} derived from
\emph{D'Alembert's principle} to describe the dynamic of the
motions. To simplify the math, let's temporarily imagine the entire
human skeleton consists of a collection of particles
$\{\vGlobalPoint_1, \vGlobalPoint_2, \ldots,
\vGlobalPoint_{\sNumParticle}\}$.  Each particle, $\vGlobalPoint_i$,
is defined by Cartesian coordinates that describe the translation with
respective to the world coordinates.  We can represent
$\vGlobalPoint_i$ by a set of \emph{generalized coordinates} that
indicate the joint configuration of the human skeleton:
\begin{equation}\label{eq:general_coord}
    \vGlobalPoint_i = \vGlobalPoint_i(\sJoint_{1},
\sJoint_{2}, \ldots, \sJoint_{\sNumJoint}, t)
\end{equation}
where $t$ is the time and $\sJoint_j$ is a joint degree of freedom (DOF) in
the skeleton.

The virtual displacement $\delta \vGlobalPoint_i$ refers to an
infinitesimal change in the system coordinates such that the
constraint remains satisfied. In the context of human skeleton, the
system coordinates are the generalized coordinates $\sJoint_j$ and the
constraint manifold lies in the Cartesian space. The virtual
displacement $\delta \vGlobalPoint_i$ is a tangent vector to the
constraint manifold at a fixed time, written as
\begin{equation}\label{eq:virtual_displace}
    \delta \vGlobalPoint_i = \sum_j \frac{\partial \vGlobalPoint_i}{\partial
    \sJoint_j}\delta \sJoint_j
\end{equation}

We can write virtual work of force $\vForce{i}$ acting on particle
$\vGlobalPoint_i$ as
\begin{equation}\label{eq:virtual_work}
  \vc{f}_{i} \cdot  \delta \vc{r}_i = \vc{f}_{i} \cdot  \sum_j \frac{\partial \vc{r}_i}{\partial
    q_j}\delta q_j \equiv \sum_j Q_{ij} \delta q_j = \vc{Q}_i \cdot \delta \vc{q}
\end{equation}
where $Q_{ij} = \left ( \frac{\partial \vc{r}_i}{\partial q_j} \right )^T \vc{f}_i$ is defined as the component of the \emph{generalized force}
associated with coordinate $q_j$. In the vector form, $\vc{Q}_i$ is the generalized force corresponding to the Cartesian force $\vc{f}_i$ with the relation $\vc{Q}_i = J_i^T \vc{f}_i$, where $J_i$ is the Jacobian matrix with the $j^{th}$ column defined as $\frac{\partial \vc{r}_i}{\partial q_j}$.

From D'Alembert's principle, we know that the sum of the differences
between the forces acting on a system and the inertia force of the
system along any virtual displacement consistent with the constraints
of the system, is zero. Therefore, the virtual work at $\vGlobalPoint_i$ can be written as
\begin{equation}\label{eq:inertial_work}
  \delta \sWork_i = \vForce{i} \cdot  \delta \vGlobalPoint_i =
  \sInfMass_i \ddot{\vGlobalPoint}_i \cdot \delta \vGlobalPoint_i =
  \sum_j \sInfMass_i \ddot{\vGlobalPoint}_i \cdot
    \frac{\partial \vGlobalPoint_i}{\partial \sJoint_j} \delta \sJoint_j
\end{equation}
where $\sInfMass_i$ is the infinitesimal mass associated with
$\vGlobalPoint_i$. The component of inertia force associated with
$\sJoint_j$ can be written as
\begin{equation}
\label{eq:inertia_force}
  \sInfMass_i \ddot{\vGlobalPoint}_i \cdot \frac{\partial \vGlobalPoint_i}{\partial \sJoint_j} =
  \frac{d}{dt} \left( \sInfMass_i \dot{\vGlobalPoint}_i \cdot \frac{\partial
  \vGlobalPoint_i}{\partial \sJoint_j} \right) - \sInfMass_i
\dot{\vGlobalPoint}_i \cdot \frac{d}{dt} \left( \frac{\partial
    \vGlobalPoint_i}{\partial \sJoint_j} \right) 
\end{equation}

Now let us consider the velocity of $\vGlobalPoint_i$ in terms of
joint velocity $\dot{\sJoint}_j$
\begin{equation}
\label{eq:velocity}
\dot{\vGlobalPoint}_i = \sum_j \frac{\partial
  \vGlobalPoint_i}{\partial \sJoint_j} \dot{\sJoint}_j
\end{equation}
from which we derive the following two identities:
\begin{eqnarray}
\frac{\partial \dot{\vGlobalPoint}_i}{\partial \dot{\sJoint}_j} &=&
\frac{\partial \vGlobalPoint_i}{\partial \sJoint_j}  \\
\frac{\partial \dot{\vGlobalPoint}_i}{\partial \sJoint_j} &=&
\sum_k \frac{\partial^2 \vGlobalPoint_i}{\partial \sJoint_j \partial
  \sJoint_k} \dot{\sJoint}_k = \frac{d}{dt} \frac{\partial \vGlobalPoint_i}{\partial \sJoint_j}
\end{eqnarray}

Using these two identities, we rewrite \eqnref{inertia_force} as
\begin{equation}
\label{eq:inertia_force2}
\sInfMass_i \ddot{\vGlobalPoint}_i \cdot \frac{\partial \vGlobalPoint_i}{\partial \sJoint_j}   = \frac{d}{dt} \left ( \frac{\partial}{\partial \dot{\sJoint}_j} \left( \frac{1}{2} \sInfMass_i \dot{\vGlobalPoint}_i^{T} \dot{\vGlobalPoint}_i\right) \right)
   - \frac{\partial}{\partial \sJoint_j} \left( \frac{1}{2} \sInfMass_i \dot{\vGlobalPoint}_i^{T} \dot{\vGlobalPoint}_i \right)
\end{equation}

We can denote the kinetic energy of $\vGlobalPoint_i$ as
\begin{equation}\label{eq:kinetic_energy}
    \sKinetics_i = \frac{1}{2} \sInfMass \dot{\vGlobalPoint}_i^{T}
    \dot{\vGlobalPoint}_i,
\end{equation}
and rewrite \eqnref{inertia_force2} as
\begin{equation}\label{eq:inertia_kinetic}
    \sInfMass_i \ddot{\vGlobalPoint}_i \cdot \frac{\partial \vGlobalPoint_i}{\partial
    \sJoint_j} = \frac{d}{dt} \left( \frac{\partial \sKinetics_i}{\partial
    \dot{\sJoint}_j}\right) - \frac{\partial \sKinetics_i}{\partial \sJoint_j}
\end{equation}

Combining the definition of generalized force (\eqnref{virtual_work}), D'Alembert's principle (\eqnref{inertial_work}), and the generalized inertia force (\eqnref{inertia_kinetic}), we arrive at the following equation:
\begin{equation}\label{eq:dynamic_equil}
    \left ( \frac{d}{dt} \left( \frac{\partial \sKinetics_i}{\partial \dot{\sJoint}_j} \right) - \frac{\partial \sKinetics_i}{\partial
    \sJoint_j}\right ) \delta \sJoint_j = Q_{ij} \delta
    \sJoint_j
\end{equation}

If the set of generalized coordinates $\sJoint_j$ is linearly
independent, \eqnref{dynamic_equil} leads to
\emph{Lagrangian equation}:
\begin{equation}\label{eq:lagrangian_dyn2}
    \frac{d}{dt} \left( \frac{\partial \sKinetics_i}{\partial
    \dot{\sJoint}_j} \right) - \frac{\partial \sKinetics_i}{\partial
    \sJoint_j} - Q_{ij} = 0
\end{equation}

\paragraph{Equations of Motion in Vector Form.} \eqnref{lagrangian_dyn2} is the equation of motion for one generalized coordinate in a
multibody system. We can combine $\sNumJoint$  scalar equations into
the familiar vector form
\begin{equation}\label{eq:lagrangian_vector}
M(\vc{q}) \ddot{\vc{q}} + C(\vc{q}, \dot{\vc{q}}) = \vc{Q} 
\end{equation}
where $M(\vc{q})$ is the mass matrix, $C(\vc{q}, \dot{\vc{q}})$ is the
Coriolis and centrifugal term of the equation of motion, and $\vc{Q}$
is the vector of generalized forces for all the degrees of freedom
(DOFs) in the system. $M$ only depends on $\vc{q}$ and $C$ depends
quadratically on $\dot{\vc{q}}$.

How do we derive $M$ and $C$ from \eqnref{lagrangian_dyn2}?
Let us go back to the velocity of one particle $\vGlobalPoint_i$:
\begin{equation}
\dot{\vGlobalPoint}_i = \sum_j \frac{\partial
  \vGlobalPoint_i}{\partial \sJoint_j} \dot{\sJoint}_j = J_i(\vc{q}) \dot{\vc{q}}
\end{equation}
where $J_i$ denotes the Jacobian of $\vGlobalPoint_i$. By summing up
all the particles in the system, the kinetic
energy of the system can then be expressed as
\begin{equation}
\label{eq:kinetic_vector}
T = \sum_i T_i = \sum_i \frac{1}{2}  \sInfMass \dot{\vGlobalPoint}_i^{T}
    \dot{\vGlobalPoint}_i = \sum_i \frac{1}{2}  \sInfMass (J_i
    \dot{\vc{q}})^T(J_i \dot{\vc{q}}) = \frac{1}{2} \dot{\vc{q}}^T
    (\sum_i \sInfMass J_i^TJ_i) \dot{\vc{q}} = \frac{1}{2}
    \dot{\vc{q}}^T M(\vc{q}) \dot{\vc{q}}
\end{equation}
where we define the mass matrix, $M(\vc{q}) = \sum_i \sInfMass
J_i^TJ_i$, and will shortly show it is indeed the mass matrix in
\eqnref{lagrangian_vector}.

From \eqnref{kinetic_vector}, we can derive the derivative
terms to construct the Lagrange's equation (\eqnref{lagrangian_dyn2}):
\begin{equation}
\label{eq:lagrangian_vector2}
\frac{d}{dt}\frac{\partial T}{\partial \dot{\vc{q}}} - \frac{\partial
  T}{\partial \vc{q}} = M\ddot{\vc{q}} + \dot{M} \dot{\vc{q}} - \frac{1}{2}\dot{\vc{q}}^T \left ( \frac{\partial M}{\partial \vc{q}} \right )^T \dot{\vc{q}} \equiv M
\ddot{\vc{q}} + C(\vc{q}, \dot{\vc{q}})
\end{equation}

Comparing \eqnref{lagrangian_vector2} to \eqnref{lagrangian_vector},
we confirm that the mass matrix is identical in both equations. $C$ is
the Coriolis and centrifugal term in \eqnref{lagrangian_vector} and is
defined as $C = \dot{M} \dot{\vc{q}} - \frac{1}{2}\left( \frac{\partial M}{\partial  \vc{q}} \dot{\vc{q}} \right)^T \dot{\vc{q}}$. 

\paragraph{Note.} In the second term of $C$, we introduce tensor notation $\frac{\partial M}{\partial \vc{q}}$, which implies that the $j^{th}$ element of the tensor $\frac{\partial M}{\partial \vc{q}}$ is the matrix $\frac{\partial M}{\partial {q}_j}$. Note that, in general, the quantity with notation $\frac{\partial M}{\partial \vc{q}} \dot{\vc{q}}$ is \textbf{\emph{not}} equal to $\dot{M}$. This is because, the $j^{th}$ column of the matrix $\frac{\partial M}{\partial \vc{q}} \dot{\vc{q}}$ is the vector $\frac{\partial M}{\partial q_j} \dot{\vc{q}}$ or $\sum_k \frac{\partial (M)_k}{\partial q_j} \dot{{q}_k}$, where the notation $(A)_j$ denotes the $j^{th}$ column of the matrix $A$. In contrast, the $j^{th}$ column of the matrix $\dot{M}$ is $\sum_k \frac{\partial (M)_j}{\partial q_k} \dot{{q}_k}$.

Once we know how to compute the mass matrix, Coriolis and centrifugal
term, and generalized forces, we can compute the acceleration in
generalized coordinates, $\ddot{\vc{q}}$, for forward
simulation. Conversely, if we are given $\ddot{\vc{q}}$ from a motion
sequence, we can use these equations of motion to derive generalized
forces for inverse dynamics. 

The above formulation is convenient for a system consisting of finite
number of mass points. However, for a dynamic system that consists of
rigid bodies, there are infinitely many points contained in each rigid
body making the above formulation intractable. In the following two
sections, we view a rigid body as a continuum and derive compact
equations of motions in both Cartesian coordinates and generalized
coordinates.

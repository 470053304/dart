\section{Introduction}
If you have not read the excellent SIGGRAPH course notes on
physics-based animation by Witkin and Baraff, you can stop reading
further right now. Go look for those notes at
{\tt http://www.cs.cmu.edu/\textasciitilde baraff/sigcourse/} and come back when
you fully understand everything in the notes.

If you are still reading this document, you probably fit the following
profile. You are a computer scientist with no mechanical engineering
background and minimal training in physics in high school but you are
seriously interested in physics-based character animation. You have
read Witkin and Baraff's SIGGRAPH course notes a few times but don't
know where to go from simulating rigid bodies to human figures. You
have played with some commercial physics engines like ODE (Open
Dynamic Engine), PhysX, Havok, or Bullet, but you wish to
simulate human behaviors more interesting than ragdoll effects.

Physics-based character animation consists of two parts: simulation
and control. This document focuses on the simulation part. It's quite
likely that you do not need to understand how underlying simulation
works if your control algorithm is simple enough. However, complex
human behaviors often require sophisticated controllers that exploit
the dynamics of a multibody system. A good understanding of multibody
dynamics is paramount for designing effective controllers.

There are many ways to learn multibody dynamics. Reading a textbook
on this topic or taking a course from the mechanical engineering
department will both do the job. However, if you only want to learn
the minimal set of multibody dynamics necessary to jump start your
research in physics-based character animation, this document might be
what you are looking for. In particular, this document attempts to
answer the following questions.

\begin{itemize}
\item I know how to derive the equations of motion for one rigid body
  and I have seen people use the following equations for articulated
  rigid bodies, but I don't know how they are derived.
\begin{equation}
M(\vc{q}) \ddot{\vc{q}} + C(\vc{q}, \dot{\vc{q}})  = \vc{Q} \nonumber
\end{equation}

\item I have seen Lagrangian equation in the following form before, but I
  don't know how it is related to the equations of motion above.
\begin{equation}\label{eq:lagrangian_dyn}
    \frac{d}{dt} \left( \frac{\partial \sKinetics_i}{\partial
    \dot{\vJoint}} \right) - \frac{\partial \sKinetics_i}{\partial
    \vJoint} - \vc{Q} = 0 \nonumber
\end{equation}

\item I use generalized coordinates to compute the control forces, how do I convert them to Cartesian forces such that I can use simulators like ODE, PhysX, or Bullet which represent rigid bodies in the maximal coordinates?

\item I heard inverse dynamics can be computed recursively to improve performance. How does that work?
\end{itemize}


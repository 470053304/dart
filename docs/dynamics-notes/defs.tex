%% editing comment

%\newcommand{\cmt}[1]{\textcolor{red}{\textbf {#1}}}
\newcommand{\cmt}[1]{}
\newcommand{\note}[1]{\cmt{Note: #1}}
\newcommand{\karen}[1]{\textcolor{red}{{Karen: #1}}}
\newcommand{\sumit}[1]{\textcolor{blue}{{Sumit: #1}}}
\newcommand{\newtext}[1]{#1}
%\newcommand{\newtext}[1]{\textcolor{blue}{#1}}
\newcommand{\eqnref}[1]{Equation~(\ref{eq:#1})}

%% ignore text
\long\def\ignorethis#1{}

%% abbreviations
\newcommand{\etal}{{\em{et~al.}\ }}
\newcommand{\eg}{e.g.\ }
\newcommand{\ie}{i.e.\ }

%% reference shortcuts
\newcommand{\figtodo}[1]{\framebox[0.8\columnwidth]{\rule{0pt}{1in}#1}}
\newcommand{\figref}[1]{Figure~\ref{fig:#1}}
%\renewcommand{\eqref}[1]{Equation~(\ref{eq:#1})}
\newcommand{\secref}[1]{Section~\ref{sec:#1}}

%% frequently used mathematical structures
\newcommand{\vc}[1]{\ensuremath{\mathbf{#1}}}
\newcommand{\pd}[2]{\ensuremath{\frac{\partial{#1}}{\partial{#2}}}}
\newcommand{\pdd}[3]{\ensuremath{\frac{\partial^2{#1}}{\partial{#2}\,\partial{#3}}}}


% math macros for phylter
\newcommand{\mTrans}{\ensuremath{\vc{T}}}
\newcommand{\vLinMom}{\ensuremath{\vc{P}}}
\newcommand{\vAngMom}{\ensuremath{\vc{L}}}
\newcommand{\vMomVar}{\ensuremath{\vc{u}}}
\newcommand{\vLine}{\ensuremath{\vc{l}}}
\newcommand{\vControlPoint}{\ensuremath{\vc{p}}}
\newcommand{\vControlKnot}{\ensuremath{\vc{u}}}

% math macros for style
\newcommand{\vStyle}{\ensuremath{\mathbf{\theta}}}
\newcommand{\sMusclePref}{\ensuremath{\alpha}}
\newcommand{\sKinetics}{\ensuremath{T}}
\newcommand{\sSpringCoef}{{\ensuremath{k_{s}}}}
\newcommand{\sDamperCoef}{{\ensuremath{k_{d}}}}
\newcommand{\sSpringCoefPos}{{\ensuremath{k_{s1}}}}
\newcommand{\sSpringCoefNeg}{{\ensuremath{k_{s2}}}}
\newcommand{\vLagrange}{\ensuremath{\vc{\lambda}}}
\newcommand{\sLagrange}{\ensuremath{\lambda}}
\newcommand{\sShoeRestLen}{\ensuremath{\bar{h}}}
\newcommand{\sShoeDist}{\ensuremath{h}}
\newcommand{\sShoeCoef}{{\ensuremath{k_{\mathit{shoe}}}}}
\newcommand{\vHandle}{\ensuremath{\vc{h}}}
\newcommand{\temp}{\ensuremath{\tau}}

%% math macros for multi-char
\newcommand{\vJointCoef}{{\ensuremath{\vc{h}_{q}}}}
\newcommand{\vTimeCoef}{\ensuremath{\vc{h}_t}}
\newcommand{\sTimeWarpFunc}{\ensuremath{F_t}}

%% math macros for phoward
%\newcommand{\vForce}{\ensuremath{\vc{f}}}
\newcommand{\sObjFunc}{\ensuremath{E}}
\newcommand{\vConeCoef}{\ensuremath{\vc{w}}}
\newcommand{\sConeCoef}{\ensuremath{w}}

%% frequently used symbols
\newcommand{\mTransChain}{\ensuremath{\vc{W}}}
\newcommand{\mRotateTrans}{\ensuremath{\vc{R}}}
\newcommand{\sJoint}{\ensuremath{q}}
\newcommand{\vJoint}{\ensuremath{\vc{q}}}
\newcommand{\mJoint}{\ensuremath{\vc{Q}}}
\newcommand{\mMass}{\ensuremath{\vc{M}}}
\newcommand{\mUnknown}{\ensuremath{\vc{X}}}
\newcommand{\sMass}{\ensuremath{{m}}}
\newcommand{\sInfMass}{\ensuremath{\mathbf{\mu}}}
\newcommand{\vGravity}{\ensuremath{\vc{g}}}
\newcommand{\vConstr}{\ensuremath{\vc{C}}}
\newcommand{\sConstr}{\ensuremath{C}}
\newcommand{\mGeometry}{\ensuremath{\vc{V}}}
\newcommand{\vCOM}{\ensuremath{\vc{c}}}
\newcommand{\sNumJoint}{\ensuremath{n_j}}
\newcommand{\sNumFrame}{\ensuremath{n_f}}
\newcommand{\sNumConstr}{\ensuremath{n_c}}
\newcommand{\sNumParticle}{\ensuremath{n_p}}
\newcommand{\vGlobalPoint}{\ensuremath{\vc{r}}}
\newcommand{\vLocalPoint}{\ensuremath{\vc{x}}}
\newcommand{\sGeneralForce}[1]{\ensuremath{Q_{#1}}}
\newcommand{\vForce}[1]{\ensuremath{\vc{f}_{#1}}}
\newcommand{\sWork}{\ensuremath{W}}
\newcommand{\vStateVar}{\ensuremath{\vc{y}}}
\newcommand{\vControlVar}{\ensuremath{\vc{u}}}
%\newcommand{\sObjFunc}{\ensuremath{F}}
\newcommand{\sCurve}{\ensuremath{S}}
\newcommand{\vCurve}{\ensuremath{\vc{S}}}
\newcommand{\argmax}{\operatornamewithlimits{argmax}}
\newcommand{\argmin}{\operatornamewithlimits{argmin}}


\newcommand{\tr}[1]{\ensuremath{\mathrm{tr}\left(#1\right)}}




%%%%%%%%%%%%%%%%%%%%%%%%%%%%%%%%%%%%%%%%%%%%%%%%%%%%%%%%%%%%%%%%%%%
%
% Here are a bunch of macros, mostly for math.
%
%%%%%%%%%%%%%%%%%%%%%%%%%%%%%%%%%%%%%%%%%%%%%%%%%%%%%%%%%%%%%%%%%%%

\renewcommand{\choose}[2]{\ensuremath{\left(\begin{array}{c} #1 \\ #2 \end{array} \right )}}

\newcommand{\gauss}[3]{\ensuremath{\mathcal{N}(#1 | #2 ; #3)}}

\newcommand{\pctab}{\hspace{0.2in}}
\newenvironment{pseudocode} {\begin{center} \begin{minipage}{\textwidth}
                             \normalsize \vspace{-2\baselineskip} \begin{tabbing}
                             \pctab \= \pctab \= \pctab \= \pctab \=
                             \pctab \= \pctab \= \pctab \= \pctab \= \\}
                            {\end{tabbing} \vspace{-2\baselineskip}
                             \end{minipage} \end{center}}
\newenvironment{items}      {\begin{list}{$\bullet$}
                              {\setlength{\partopsep}{\parskip}
                                \setlength{\parsep}{\parskip}
                                \setlength{\topsep}{0pt}
                                \setlength{\itemsep}{0pt}
                                \settowidth{\labelwidth}{$\bullet$}
                                \setlength{\labelsep}{1ex}
                                \setlength{\leftmargin}{\labelwidth}
                                \addtolength{\leftmargin}{\labelsep}
                                }
                              }
                            {\end{list}}
\newcommand{\newfun}[3]{\noindent\vspace{0pt}\fbox{\begin{minipage}{3.3truein}\vspace{#1}~ {#3}~\vspace{12pt}\end{minipage}}\vspace{#2}}



\newcommand{\key}{\textbf}
\newcommand{\fun}{\textsc}

\def\shortcite{\def\citename##1{}\@internalcite}


% Local Variables:
% TeX-master: "paper"
% End:

\section{Forward Simulation}
In this section, we will build and simulate a simple chain of rigid
bodies connected by ball joints. This example demonstrates how an
articulated rigid body system is represented using DART data
structures and how a simulation step is formulated using different
numerical integration methods.

Because this is the first example, we will examine the source code in
more details. To create and visualize a basic simulation app, we
integrate DART into Glut framework. All the examples in this tutorial
use Glut and OpenGL libraries to handle user interface and
rendering. However, DART does not assume a particular UI or rendering
package. 


\subsection{Source code}
Let us begin with the main function. The main function does two
things: loading a skeleton file and creating a window for UI and
rendering. 

A skeleton comprises a set of body nodes connected by
joints. We can directly load a skeleton from a file or manually build a new
skeleton in the code. DART currently supports two formats of skeleton
file: .vsk and .skel. In this example, we load in a skeleton with 10
rigid links connected by ball joints. 

Once \textbf{glutMainLoop} is invoked in the main function, Glut
callback functions will start handling display updates, keyboard
inputs, etc. The simulation code is called from the callback function
\textbf{displayTimer()} in MyWindow.cpp.

\ttfamily
\begin{lstlisting}[label=displayTimer,caption=displayTimer]
void MyWindow::displayTimer(int _val)
{
    int numIter = mDisplayTimeout / (1000 * mTimeStep);
    for (int i = 0; i < numIter; i++) {
      mIntegrator.integrate(this, mTimeStep);
      mFrame++;
    }
    glutPostRedisplay();
    if (mRunning)	
        glutTimerFunc(mDisplayTimeout, refreshTimer, _val);
}
\end{lstlisting}
\rmfamily
\textbf{displayTimer()} is called every \textbf{mDsiplayTimeout}
millisecond. Because \textbf{mDsiplayTimeout} is typically larger than
the simulation time step (\textbf{mTimeStep}), we need to run a few
iterations of simulation steps each time \textbf{displayTimer()} is
called. The number of iterations is computed as
\textbf{numIter}. \textbf{mIntegrator} is a data member of
\textbf{MyWindows} and can be declared as an explicit Euler integrator
or a RK4 integrator in MyWindows.h.

\subsection{DART libraries}

\paragraph{Skeleton Class:}

\paragraph{BodyNode Class:}

\paragraph{Joint Class:}

\section{Introduction}
DART is an open source library for developing kinematic and dynamic
applications in robotics and computer animation. DART features two
main components: a multibody dynamic simulator developed by Georgia
Tech Graphics Lab and a variety of motion planning algorithms
developed by Georgia Tech Humanoid Robotics Lab. This document focuses
only on the dynamic simulator.
 
The multibody dynamics simulator in DART is designed to aid
development of motion control algorithms. DART uses generalized
coordinates to represent articulated rigid body systems and computes
Lagrange’s equations derived from D’Alembert’s principle to describe
the dynamics of motion. In contrast to many popular physics engines
which view the simulator as a black box, DART provides full access to
internal kinematic and dynamic quantities, such as mass matrix,
Coriolis and centrifugal force, transformation matrices and their
derivatives, etc. DART also provides efficient computation of Jacobian
matrices for arbitrary body points and coordinate frames.

The contact and collision are handled using an implicit time-stepping,
velocity-based LCP (linear-complementarity problem) to guarantee
non-penetration, directional friction, and approximated Coulombs
friction cone conditions. The LCP problem is solved efficiently by
Lemke's algorithm. For the collision detection, DART directly uses FCL
package developed by UNC Gamma Lab.

In addition, DART supports various joint types (ball-and-socket,
universal, hinge, and prismatic joints) and arbitrary meshes. DART
also provides two explicit integration methods: first-order
Runge-Kutta and fourth-order Runge Kutta.

The rest of tutorial will go over 7 examples included in the source
code. These examples demonstrate how to use DART for developing
forward simulation, inverse kinematics, inverse dynamics, open-loop
and feedback control for a manipulator, underactuated balance control
for a biped system, and hybrid control for a mobile manipulator.
